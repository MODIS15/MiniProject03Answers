\section{Scalability}
\textbf{\large{Handling clustering of resources}}\\
The resources in the Nodes are never redistributed to create an even allocation of the resources, which means that the network can end up with big clusters of resources in single Nodes. This could have been avoided by having some kind of capacity indicator for each Node. When a resource is PUT in a Node, it could then ask its neighbors if they had less resources than itself. If this is the case, the resource would be passed on to this Node. This would continue, until arriving at a Node with less resources than its neighbors or until the resource arrives at the initial Node. In this way the resources would be somewhat evenly distributed across the network.\\\\
\textbf{\large{Handling GET message for none existing resource}}\\
If a request comes in asking for a key which does not exist, the GET message will go around forever. This constitutes a scalability problem if one assume that the users of the network does not always ask for a key in the network. A possible scenario could be an adversary who wants to crash the network and therefor generates an arbitrarily high number of messages asking for non existing resources and eventually the number of messages going around in the network will make it crash.
This could have been avoid by letting the GET message know if it had already visited a Node, for instance by saving some of the Node information in the message when the Node is requested. This would mean that the message would be dropped after asking the last Node for the resource.\\\\
\textbf{\large{Better routing}}\\
The current network is unstructured which makes the searching for resources less efficient than if the network had been structured. Since a ring topology has been used to create the network an obvious choice would be to use the Pastry overlay, which would have improved the routing by the use of a distributed hash table.